Developers work in an immense online network.
Developers benefit from this network by
leveraging social media and other channels to stay informed
and connect with other developers~\cite{singer_software_2014,storey_how_2016}.
Today there is a proliferation of socially-enabled channels~\cite{storey_revolution_2014} through which developers
answer each other's questions~\cite{mamykina_fastest_2011},
stay up-to-speed with rapidly changing software~\cite{linares-vasquez_how_2014},
help each other overcome bugs and learn new tools~\cite{parnin_blogging_2013},
and share information in many different forms at many different speeds.
The knowledge and conversations of programmers are increasingly distributed across the web.

Some community-developed tools aim to help developers compare packages.
Such community tools only show a fraction of the metrics used by developers.
Typical metrics for these tools (e.g.,~\cite{awesome_python,package_quality,ruby_toolbox}) tend to be based on counts and rates of code contributions, issue resolutions, downloads, and ``stars'' on code hosting sites like GitHub.
Developers may be concerned with
how much they respect the authors of code~\cite{robillard_field_2011},
how up-to-date the documentation is~\cite{lethbridge_how_2003,nykaza_what_2002,robillard_field_2011,storey_revolution_2014},
and whether the community is anti-social~\cite{storey_revolution_2014}.
We propose that carefully selected samples from communication channels can help developers make more informed judgments based on social health.
The current work presents a study to help us pick these samples.
