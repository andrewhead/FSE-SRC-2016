We invited 10 developers to a study where they compared the social health of two packages.
Each participant chose a pair of packages, out of three pairs.
Then we asked each participant to compare the two packages, using six questions about each package's community, documentation, and developers.
The questions were based on related work and conversations we had with software developers, and included:
\begin{itemize}
\setlength{\itemsep}{0pt}
\setlength{\parskip}{0pt}
\setlength{\parsep}{0pt}  
\item Which community will be more welcoming when responding to questions you ask?
\item Which package's documentation will be more up-to-date with the code?
\item Which package's developers can you better trust to make reliable, usable software?
\end{itemize}
These three questions (out of six) were based on observations of anti-social behavior in developers' communication channels~\cite{storey_revolution_2014}, concerns about how up-to-date documentation is~\cite{lethbridge_how_2003,nykaza_what_2002,robillard_field_2011,storey_revolution_2014}, and how developers assess code example quality based on author reputation~\cite{robillard_field_2011}.

Participants answered the questions using the web.
We note developers often get answers face-to-face~\cite{latoza_maintaining_2006,storey_revolution_2014}, over email~\cite{ko_information_2007,latoza_maintaining_2006}, or by trying out code~\cite{brandt_two_2009}.
However, we focused on information on the web, as we believed this would reveal signals we could incorporate into future search tools.

We collected three measurements of cues participants used to answer social questions:
a timestamped log of visited URLs;
self-reported ratings of web pages' ``helpfulness'' for answering each question;
and open-ended responses of what evidence on the web was most helpful for comparing the social health of the two packages for each question.
We report a view of our current results here.
Our on-going analysis has focused on two research questions:
(1) What sites do participants use to learn about a package's social health?
(2) What challenges do developers face when learning about a package's social health?
